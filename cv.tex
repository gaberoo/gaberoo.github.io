\documentclass[11pt,a4paper]{article}

\usepackage[top=2cm,bottom=2cm,left=4cm,right=2cm]{geometry}

\usepackage{amsmath,amssymb,amsfonts}
\usepackage{color}
\usepackage[T1]{fontenc}

\PassOptionsToPackage{hyphens}{url}
\usepackage[hidelinks=true,bookmarks=true,citecolor=red]{hyperref}
\renewcommand\UrlFont{{\condensedbf}}

% FONTS %%%%%%%%%%%%%%%%%%%%%%%%%%%%%%%%%%%%%%%%%%%%%%%%%%%%%%%%%%%%%%%%%%%%%%

\usepackage[hypcap=true,font={small,sf},labelfont={bf}]{caption}
\usepackage{mathspec}
\defaultfontfeatures{Ligatures=TeX}

\setmainfont{Adobe Caslon Pro}
\setsansfont[%
BoldFont=*-Medium,
UprightFont=*-Regular,
BoldItalicFont=*-SemiBold Italic
]{Barlow}
\setmathfont(Digits,Latin)[Scale=MatchLowercase]{Adobe Caslon Pro}

\newfontfamily\condensed[Ligatures=TeX]{Barlow Condensed Regular}
\newfontfamily\semicond[Ligatures=TeX]{Barlow Semi Condensed Regular}
\newfontfamily\condensedbf[Ligatures=TeX]{Barlow Condensed Medium}
\newfontfamily\boldsf[Ligatures=TeX]{Barlow Semi Condensed SemiBold}

\usepackage{titlesec}
\titleformat*{\section}{\Large\boldsf\titlerule}
\titleformat*{\subsection}{\large\boldsf\em}
\titleformat*{\subsubsection}{\boldsf}

\usepackage{caption}
\captionsetup[figure]{labelfont={bf,sf},textfont=sf}
\captionsetup[table]{labelfont={bf,sf},textfont=sf}

\usepackage{enumitem}
\setlist{noitemsep} 

\usepackage{etaremune}

\usepackage{mdframed}

\mdfdefinestyle{hlBox}{%
  hidealllines=true,
  leftline=true,
  linecolor=red,
  leftmargin=-4.5em,
  rightmargin=-1.5em,
  linewidth=0.5em,
  outerlinewidth=2pt,
  innertopmargin=4pt,
  innerbottommargin=4pt,
  innerrightmargin=1em,
  innerleftmargin=2em,
}

%%%%%%%%%%%%%%%%%%%%%%%%%%%%%%%%%%%%%%%%%%%%%%%%%%%%%%%%%%%%%%%%%%%%%%%%%%%%%%

\usepackage{marginnote} % For margin years
\newcommand{\margin}[1]{\marginnote{\raggedleft\sf\scriptsize #1}}
\renewcommand*{\raggedleftmarginnote}{}
\setlength{\marginparsep}{5pt}
\setlength{\marginparwidth}{4cm}
\reversemarginpar

%%%%%%%%%%%%%%%%%%%%%%%%%%%%%%%%%%%%%%%%%%%%%%%%%%%%%%%%%%%%%%%%%%%%%%%%%%%%%%

\newcommand{\doi}[1]{\href{https://dx.doi.org/#1}{#1}}
\newcommand{\journal}[1]{{\condensedbf{#1}}}
\newcommand{\journalissue}[1]{{\sffamily{#1}}}
\newcommand{\pubyear}[1]{(#1)}
\newcommand{\me}[1]{\underline{\boldsf{#1}}}
\newcommand{\pubtitle}[1]{\textit{#1}}
\newcommand{\pubpages}[1]{#1}
\newcommand{\contribution}[1]{\\{\sffamily(\textit{#1})}}

\setlength\parindent{0pt}

\begin{document}

%%Header
\sffamily

{\small\boldsf Curriculum Vitae}\\
{\LARGE\boldsf Gabriel Etan Leventhal}

\medskip

\begin{minipage}[t]{.5\textwidth}
  Forch 2\\ 
8574 Oberhofen\\ 
Switzerland\\ 

\end{minipage}%
\begin{minipage}[t]{.5\textwidth}
  Phone: +41 (78) 708 49 08 \\
  Email: gabriel@leventhal.ch \\
  URL: \url{http://www.leventhal.ch} \\
\end{minipage}

\textbf{Language Skills:} English (native), Swiss German (native), German (fluent), French (fluent)

\section*{Current Position}
\margin{%
  Mar 2020
  ---
  present
}

{\bfseries VP Data Science}%
\\ Pharmabiome AG%

\medskip



\section*{Areas of Specialization}
Microbial Ecosystems; Evolutionary Theory; Interaction Networks; Community Evolution; Evolutionary Ecology; Computational Biology; Phylogenetics; Statistical Inference; Ecological Networks; Epidemiology; HIV Dynamics

\section*{Academic Positions and Education}
  \margin{%
  Nov 2015
  ---
  Feb 2020
}

{\bfseries Postdoctoral Fellow, Massachusetts Institute of Technology (MIT), U.S.A.}%
\\ Cordero Lab, Department of Civil and Environmental Engineering%

\medskip


  \margin{%
  Sep 2014
  ---
  Oct 2015
}

{\bfseries Postdoctoral Researcher, ETH Zürich, Switzerland}%
\\ Theoretical Biology Group, Institute of Integrative Biology%

\medskip


  \margin{%
  Sep 2009
  ---
  Aug 2014
}

{\bfseries PhD Thesis, ETH Zürich, Switzerland}%
\\ Title: Modeling the ecology and evolution of infectious diseases\\ Supervisor: Prof. Sebastian Bonhoeffer

\medskip


  \margin{%
  Mar 2008
  ---
  Jun 2008
}

{\bfseries Research Assistant, EPFL, Lausanne, Switzerland}%
\\ Laboratory of Statistical Biophysics, Department of Physics%

\medskip


  \margin{%
  Oct 2006
  ---
  Feb 2008
}

{\bfseries MSc in Physics, EPFL, Lausanne, Switzerland}%

\medskip


  \margin{%
  Aug 2007
  ---
  Feb 2008
}

{\bfseries Master’s Thesis, Indiana University Bloomington, U.S.A.}%
\\ Title: Spectral Coarse Graining in Ising Spin Systems\\ Supervisors: Prof. Alessandro Flammini (IUB) and Prof. Paolo De Los Rios (EPFL)

\medskip


  \margin{%
  May 2006
  ---
  Aug 2006
}

{\bfseries Research Assistant, Hong Kong University of Science and Technology}%
\\ Kwok Yip Szeto Group, Department of Physics%

\medskip


  \margin{%
  Aug 2002
  ---
  May 2006
}

{\bfseries BSc in Physics, EPFL, Lausanne, Switzerland}%

\medskip


  \margin{%
  Sep 2005
  ---
  May 2006
}

{\bfseries Undergraduate Exchange, Hong Kong University of Science and Technology}%
\\ One-year international exchange program as part of BSc in physics

\medskip



\pagebreak

\section*{Publications}

  I have published 22 original research articles in peer-reviewed journals 
(10 as first, co-first, or equal contribution, as well as 
2 review articles. Selected important publications are followed
by a brief summary in red font.

$\dagger$ equal contribution; $\ast$ student advisee

    \subsection*{Microbial Ecology \& Evolution}
    \begin{itemize}[label={},leftmargin=0pt]
        \item%
          
  \medskip

\margin{%
    2020
}%
{\semicond LS Bittleston\textsuperscript{}, M Gralka\textsuperscript{}, \me{GE Leventhal}\textsuperscript{}, I Mizrahi\textsuperscript{}, OX Cordero\textsuperscript{}}.
\pubtitle{Context-dependent dynamics lead to the assembly of functionally distinct pitcher-plant microbiomes.}
  \journal{Nature Communications} \journalissue{}
  \href{https://doi.org/10.1038/s41467-020-15169-0}{\condensed10.1038/s41467-020-15169-0}

\textit{\textcolor{red}{In this paper, I contributed to the analysis and interpretation of 
the compositional time series of ten distinct natural microbial that 
were propagated over weeks in controlled laboratory conditions. We show 
that while some aspects of the dynamics are conserved across starting 
communities, the intial composition matters for the dynamics in the same
laboratory conditions. This paper sets the ground work for a follow-up
experiment that I lead, in which we continued to propagate the 
communities during a whole year to measure the evolutinary dynamics over
extended periods of time.
}}
\medskip




        \item%
          
  \medskip

\margin{%
    2019
}%
{\semicond \me{GE Leventhal}\textsuperscript{}, M Ackermann\textsuperscript{}, K Schiessl\textsuperscript{}}.
\pubtitle{Why microbes secrete molecules to modify their environment: the case of iron-chelating siderophores.}
  \journal{Journal of the Royal Society Interface} \journalissue{16(150)}
  \href{https://doi.org/10.1098/rsif.2018.0674}{\condensed10.1098/rsif.2018.0674}

\textit{\textcolor{red}{This paper addresses the question as to why some microbes secrete 
compounds to aid in foraging for resources. Because such compounds 
become public goods, they have been extensively discussed from the 
point of view of cheating and competition. Here I quantified what the
benefits to a microbe of secretion versus keeping such compounds 
for example membrane-bound when resources are aggregated in large clumps.
My results have important consequences for understanding the length 
scale of competition and cooperation in communities.
}}
\medskip




        \item%
          
  \medskip

\margin{%
    2018
}%
{\semicond \me{GE Leventhal}\textsuperscript{}, C Boix\textsuperscript{$\ast$}, U Kuechler\textsuperscript{}, T Enke\textsuperscript{}, E Sliwerska\textsuperscript{}, C Holliger\textsuperscript{}, OX Cordero\textsuperscript{}}.
\pubtitle{Strain-level diversity drives alternative community types in millimetre-scale granular biofilms.}
  \journal{Nature Microbiology} \journalissue{3:1295–1303}
  \href{https://doi.org/10.1038/s41564-018-0242-3}{\condensed10.1038/s41564-018-0242-3}

\textit{\textcolor{red}{This paper is one of the first to investigate the fine-scale population
genomic strucutre across replicate microbial communities. By leveraging
the milimeter-scale granular biofilms from an activate granular sludge
reactor, I showed that the dominant organism maintained clear and 
diverse genetic structure across replicate communities, and that this
structure was correlated with the overall community composition. This 
is of large important to understanding the role of strain diversity and
specificity in microbial community assembly.
}}
\medskip




        \item%
          \margin{%
    2018
}%
{\semicond TN Enke\textsuperscript{}, \me{GE Leventhal}\textsuperscript{}, M Metzger\textsuperscript{}, J Saavedra\textsuperscript{}, OX Cordero\textsuperscript{}}.
\pubtitle{Microscale ecology regulates particulate organic matter turnover in model marine microbial communities.}
  \journal{Nature Communications} \journalissue{}
  \href{https://doi.org/10.1038/s41467-018-05159-8}{\condensed10.1038/s41467-018-05159-8}



        \item%
          \margin{%
    preprint
}%
{\semicond \me{GE Leventhal}\textsuperscript{$\dagger$}, L Wang\textsuperscript{$\dagger$}, RD Kouyos\textsuperscript{}}.
\pubtitle{Real-world Interaction Networks Buffer Impact of Small Evolutionary Shifts On Biodiversity.}
  \journal{bioRxiv} \journalissue{}
  \href{https://doi.org/10.1101/013086}{\condensed10.1101/013086}



    \end{itemize}
    \subsection*{Microbiome \& Health}
    \begin{itemize}[label={},leftmargin=0pt]
        \item%
          \margin{%
    preprint
}%
{\semicond E Katkeviciute\textsuperscript{}, L Hering\textsuperscript{}, A Montalban-Arques\textsuperscript{}, P Busenhart\textsuperscript{}, JC Aranda\textsuperscript{}, K Atrott\textsuperscript{}, S Lang\textsuperscript{}, G Rogler\textsuperscript{}, E Naschberger\textsuperscript{}, VS Schellerer\textsuperscript{}, M Stürzl\textsuperscript{}, A Rickenbacher\textsuperscript{}, M Turina\textsuperscript{}, A Weber\textsuperscript{}, S Leibl\textsuperscript{}, \me{GE Leventhal}\textsuperscript{}, M Levesque\textsuperscript{}, O Boyman\textsuperscript{}, M Scharl\textsuperscript{}, MR Spalinger\textsuperscript{}}.
\pubtitle{Targeting protein tyrosine phosphatase non-receptor type 2 in immune cells converts immune-silent into highly immunogenic tumors.}
  \journal{Submitted}



        \item%
          \margin{%
    2019
}%
{\semicond MR Spalinger\textsuperscript{}, M Schwarzfischer\textsuperscript{}, L Hering\textsuperscript{}, A Shawki\textsuperscript{}, A Sayoc\textsuperscript{}, A Santos\textsuperscript{}, C Gottier\textsuperscript{}, S Lang\textsuperscript{}, K Bäbler\textsuperscript{}, A Geirnaert\textsuperscript{}, C Lacroix\textsuperscript{}, \me{GE Leventhal}\textsuperscript{}, X Dai\textsuperscript{}, D Rawlings\textsuperscript{}, AA Chan\textsuperscript{}, G Rogler\textsuperscript{}, DF McCole\textsuperscript{}, M Scharl\textsuperscript{}}.
\pubtitle{Loss of PTPN22 abrogates the beneficial effect of cohousing-mediated fecal microbiota transfer in murine colitis.}
  \journal{Mucosal Immunology} \journalissue{}
  \href{https://doi.org/10.1038/s41385-019-0201-1}{\condensed10.1038/s41385-019-0201-1}



    \end{itemize}
    \subsection*{Pathogen Evolution}
    \begin{itemize}[label={},leftmargin=0pt]
        \item%
          \margin{%
    2017
}%
{\semicond F Bertels\textsuperscript{}, A Marzel\textsuperscript{}, \me{GE Leventhal}\textsuperscript{}, V Mitov\textsuperscript{}, J Fellay\textsuperscript{}, HF Günthard\textsuperscript{}, J Böni\textsuperscript{}, S Yerly\textsuperscript{}, T Klimkait\textsuperscript{}, V Aubert\textsuperscript{}, M Battegay\textsuperscript{}, A Rauch\textsuperscript{}, M Cavassini\textsuperscript{}, A Calmy\textsuperscript{}, E Bernasconi\textsuperscript{}, P Schmid\textsuperscript{}, A Scherrer\textsuperscript{}, V Müller\textsuperscript{}, S Bonhoeffer\textsuperscript{}, RD Kouyos\textsuperscript{}, RR Regoes\textsuperscript{}}.
\pubtitle{Dissecting HIV Virulence: Heritability Of Setpoint Viral Load, CD4+ T Cell Decline And Per-Parasite Pathogenicity.}
  \journal{Molecular Biology and Evolution} \journalissue{35(1):27-37}
  \href{https://doi.org/10.1093/molbev/msx246}{\condensed10.1093/molbev/msx246}



        \item%
          \margin{%
    2017
}%
{\semicond N Bachmann\textsuperscript{}, T Turk\textsuperscript{}, C Kadelka\textsuperscript{}, A Marzel\textsuperscript{}, M Shilaih\textsuperscript{}, J Böni\textsuperscript{}, V Aubert\textsuperscript{}, T Klimkait\textsuperscript{}, \me{GE Leventhal}\textsuperscript{}, HF Günthard\textsuperscript{}, RD Kouyos\textsuperscript{}}.
\pubtitle{Parent-offspring regression to estimate the heritability of an HIV-1 trait in a realistic setup.}
  \journal{Retrovirology} \journalissue{14(33)}
  \href{https://doi.org/10.1186/s12977-017-0356-3}{\condensed10.1186/s12977-017-0356-3}



        \item%
          \margin{%
    2016
}%
{\semicond \me{GE Leventhal}\textsuperscript{}, S Bonhoeffer\textsuperscript{}}.
\pubtitle{Potential pitfalls in estimating viral load heritability.}
  \journal{Trends in Microbiology} \journalissue{24(9):687-698}
  \href{https://doi.org/10.1016/j.tim.2016.04.008}{\condensed10.1016/j.tim.2016.04.008}



        \item%
          \margin{%
    2015
}%
{\semicond S Bonhoeffer\textsuperscript{}, C Fraser\textsuperscript{}, \me{GE Leventhal}\textsuperscript{}}.
\pubtitle{Heritability and the distribution of set point viral load in HIV carriers.}
  \journal{PLoS Pathogens} \journalissue{11(2):e1004634}
  \href{https://doi.org/10.1371/journal.ppat.1004634}{\condensed10.1371/journal.ppat.1004634}



        \item%
          \margin{%
    2014
}%
{\semicond C Fraser\textsuperscript{}, K Lythgoe\textsuperscript{}, \me{GE Leventhal}\textsuperscript{}, G Shirreff\textsuperscript{}, TD Hollingsworth\textsuperscript{}, S Alizon\textsuperscript{}, S Bonhoeffer\textsuperscript{}}.
\pubtitle{Virulence and Pathogenesis of HIV-1 Infection: An Evolutionary Perspective.}
  \journal{Science} \journalissue{343(6177):1243727}
  \href{https://doi.org/10.1126/science.1243727}{\condensed10.1126/science.1243727}



        \item%
          \margin{%
    2014
}%
{\semicond \me{GE Leventhal}\textsuperscript{$\dagger$}, SR Dünner\textsuperscript{$\dagger$,$\ast$}, S Barribeau\textsuperscript{}}.
\pubtitle{Delayed virulence and limited costs promote fecundity compensation upon infection.}
  \journal{American Naturalist} \journalissue{103(4):480-493}
  \href{https://doi.org/10.1086/675242}{\condensed10.1086/675242}



        \item%
          \margin{%
    2013
}%
{\semicond A Hool\textsuperscript{$\dagger$,$\ast$}, \me{GE Leventhal}\textsuperscript{$\dagger$}, S Bonhoeffer\textsuperscript{}}.
\pubtitle{Virus-induced target cell activation reconciles set-point viral load heritability and within-host evolution.}
  \journal{Epidemics} \journalissue{7:35-35}
  \href{https://doi.org/10.1016/j.epidem.2013.09.002}{\condensed10.1016/j.epidem.2013.09.002}



        \item%
          \margin{%
    2012
}%
{\semicond RD Kouyos\textsuperscript{}, \me{GE Leventhal}\textsuperscript{}, T Hinkley\textsuperscript{}, M Haddad\textsuperscript{}, J Whitcomb\textsuperscript{}, C Petropoulos\textsuperscript{}, S Bonhoeffer\textsuperscript{}}.
\pubtitle{Exploring the Complexity of the HIV-1 Fitness Landscape.}
  \journal{PLoS Genetics} \journalissue{8(3):e1002551}
  \href{https://doi.org/10.1371/journal.pgen.1002551}{\condensed10.1371/journal.pgen.1002551}



    \end{itemize}
    \subsection*{Network Epidemiology}
    \begin{itemize}[label={},leftmargin=0pt]
        \item%
          \margin{%
    2017
}%
{\semicond JI Liechti\textsuperscript{}, \me{GE Leventhal}\textsuperscript{}, S Bonhoeffer\textsuperscript{}}.
\pubtitle{Host population structure impedes reversion to drug sensitivity after discontinuation of treatment.}
  \journal{PLoS Computational Biology} \journalissue{13(8):e1005704}
  \href{https://doi.org/10.1371/journal.pcbi.1005704}{\condensed10.1371/journal.pcbi.1005704}



        \item%
          \margin{%
    2016
}%
{\semicond W Delva\textsuperscript{}, \me{GE Leventhal}\textsuperscript{}, S Helleringer\textsuperscript{}}.
\pubtitle{Connecting the dots: network data and models in HIV epidemiology.}
  \journal{AIDS} \journalissue{30(13):2009-2020}
  \href{https://doi.org/10.1097/QAD.0000000000001184}{\condensed10.1097/QAD.0000000000001184}



        \item%
          
  \medskip

\margin{%
    2015
}%
{\semicond \me{GE Leventhal}\textsuperscript{$\dagger$}, AL Hill\textsuperscript{$\dagger$}, M Nowak\textsuperscript{}, S Bonhoeffer\textsuperscript{}}.
\pubtitle{Evolution and emergence of infectious diseases in theoretical and real-world networks.}
  \journal{Nature Communications} \journalissue{6}
  \href{https://doi.org/10.1038/ncomms7101}{\condensed10.1038/ncomms7101}

\textit{\textcolor{red}{I extended the mathematical network framework commonly used in
epidemiology to include an evolutionary process. Networks are often
used to describe the precise way in which individuals in a host
population encounter each other and transmit disease. I the field of
network epidemiology, it is well known that the structure of these
networks—for example the presence of super spreaders—strongly determines
the likelihood and speed of an epidemic outbreak. How this contact
structure influences the evolution of an infectious disease, however,
had only been poorly understood. Here, I showed that the properties of
host contact networks that increase the probability and speed of the
spread of a single disease strain, also slow down the evolution of the
disease. We have identified a first-come-first-serve effect, where
host populations that are connected in such a way that facilitates the
initial emergence of disease strain, also decrease the probability that
the disease strain will be replaced by another one.
}}
\medskip




        \item%
          
  \medskip

\margin{%
    2012
}%
{\semicond \me{GE Leventhal}\textsuperscript{}, RD Kouyos\textsuperscript{}, T Stadler\textsuperscript{}, VV Wyl\textsuperscript{}, S Yerly\textsuperscript{}, J Böni\textsuperscript{}, C Cellerai\textsuperscript{}, T Klimkait\textsuperscript{}, HF Günthard\textsuperscript{}, S Bonhoeffer\textsuperscript{}}.
\pubtitle{Inferring Epidemic Contact Structure from Phylogenetic Trees.}
  \journal{PLoS Computational Biology} \journalissue{8(3):e1002413}
  \href{https://doi.org/10.1371/journal.pcbi.1002413}{\condensed10.1371/journal.pcbi.1002413}

\textit{\textcolor{red}{In this paper, I developed a mathematical approach to infer epidemic
contact struture from phylogenetic trees. Contact structure is known to 
strongly influence the spread of an infectious diesease, but measuring
contact structure is often difficult or impossible. Here, I develop a
statistical technique to infer population structure based on the shape
of the phylogenetic tree reconstructed from genomic data.
}}
\medskip




    \end{itemize}
    \subsection*{Mathematical Modelling \& Statistical Inference}
    \begin{itemize}[label={},leftmargin=0pt]
        \item%
          \margin{%
    2019
}%
{\semicond TG Vaughan\textsuperscript{$\dagger$}, \me{GE Leventhal}\textsuperscript{$\dagger$}, DA Rasmussen\textsuperscript{}, AJ Drummond\textsuperscript{}, D Welch\textsuperscript{}, T Stadler\textsuperscript{}}.
\pubtitle{Estimating epidemic incidence and prevalence from genomic data.}
  \journal{Molecular Biology and Evolution} \journalissue{36(8):1804–1816}
  \href{https://doi.org/10.1093/molbev/msz106}{\condensed10.1093/molbev/msz106}



        \item%
          \margin{%
    2017
}%
{\semicond O Ratmann\textsuperscript{}, EB Hodcroft\textsuperscript{}, M Pickles\textsuperscript{}, A Cori\textsuperscript{}, M Hall\textsuperscript{}, S Lycett\textsuperscript{}, C Colijn\textsuperscript{}, B Dearlove\textsuperscript{}, X Didelot\textsuperscript{}, S Frost\textsuperscript{}, A Hossain\textsuperscript{}, JB Joy\textsuperscript{}, M Kendall\textsuperscript{}, D Kühnert\textsuperscript{}, \me{GE Leventhal}\textsuperscript{}, R Liang\textsuperscript{}, G Plazzotta\textsuperscript{}, AF Poon\textsuperscript{}, DA Rasmussen\textsuperscript{}, T Stadler\textsuperscript{}, E Volz\textsuperscript{}, C Weis\textsuperscript{}, AJ Leigh Brown\textsuperscript{}, C Fraser\textsuperscript{}}.
\pubtitle{Phylogenetic Tools for Generalized HIV-1 Epidemics: Findings from the PANGEA-HIV Methods Comparison.}
  \journal{Molecular Biology and Evolution} \journalissue{34(1):185-203}
  \href{https://doi.org/10.1093/molbev/msw217}{\condensed10.1093/molbev/msw217}



        \item%
          \margin{%
    2016
}%
{\semicond L du Plessis\textsuperscript{}, \me{GE Leventhal}\textsuperscript{}, S Bonhoeffer\textsuperscript{}}.
\pubtitle{How good are statistical models at approximating complex fitness landscapes?.}
  \journal{Molecular Biology and Evolution} \journalissue{33(9):2454-2468}
  \href{https://doi.org/10.1093/molbev/msw097}{\condensed10.1093/molbev/msw097}



        \item%
          \margin{%
    2015
}%
{\semicond T Stadler\textsuperscript{}, TG Vaughan\textsuperscript{}, A Gavryushkin\textsuperscript{}, S Guindon\textsuperscript{}, D Kühnert\textsuperscript{}, \me{GE Leventhal}\textsuperscript{}, AJ Drummond\textsuperscript{}}.
\pubtitle{How well can the exponential-growth coalescent approximate constant-rate birth–death population dynamics?.}
  \journal{Proceeding of the Royal Society B: Biological Sciences} \journalissue{282(1806):20150420}
  \href{https://doi.org/10.1098/rspb.2015.0420}{\condensed10.1098/rspb.2015.0420}



        \item%
          \margin{%
    2014
}%
{\semicond \me{GE Leventhal}\textsuperscript{}, H Günthard\textsuperscript{}, S Bonhoeffer\textsuperscript{}, T Stadler\textsuperscript{}}.
\pubtitle{Using an epidemiological model for phylogenetic inference reveals density-dependence in HIV transmission.}
  \journal{Molecular Biology and Evolution} \journalissue{31(1):6-17}
  \href{https://doi.org/10.1093/molbev/mst172}{\condensed10.1093/molbev/mst172}



    \end{itemize}
    \subsection*{Software \& Other Publications}
    \begin{itemize}[label={},leftmargin=0pt]
        \item%
          \margin{%
    2013-2016
}%
{\semicond \me{GE Leventhal}\textsuperscript{}}.
\pubtitle{R package expoTree to calculate the density dependent likelihood of a phylogenetic tree. Available on CRAN..}



        \item%
          \margin{%
    2012-2019
}%
{\semicond \me{GE Leventhal}\textsuperscript{}, L Schulé\textsuperscript{}, J Geering\textsuperscript{}}.
\pubtitle{iRiSS: a free online journal TOC aggregator that helps you stay informed about the latest work produced in your field.}



    \end{itemize}

\pagebreak

\section*{Teaching and Mentoring}
  \margin{%
  2016, 2018
}
Co-Instructor%
,  Computational Ecology
. MIT, Cambridge, MA, USA


  \margin{%
  2018
}
Co-Supervisor%
,  Exchange graduate student
   (Jacob Russel)
. MIT, Cambridge, MA, USA


  \margin{%
  2016
}
Co-Supervisor%
,  Rotation student
   (Carles Boix)
. MIT, Cambridge, MA, USA


  \margin{%
  2013
    --- 2015
}
Co-Lecturer%
,  Infectious Disease Dynamics
. ETH Zürich, Switzerland


  \margin{%
  2015
}
Co-Supervisor%
,  Master's thesis
   (Adriano Pagano)
. ETH Zürich, Switzerland


  \margin{%
  2015
}
Supervisor%
,  Semester student
   (Adriano Pagano)
. ETH Zürich, Switzerland


  \margin{%
  2015
}
Co-Supervisor%
,  Master's thesis
   (Martin Müller)
. ETH Zürich, Switzerland


  \margin{%
  2012
}
Co-Supervisor%
,  Master's thesis
   (Anna Hool)
. ETH Zürich, Switzerland


  \margin{%
  2012
}
Co-Supervisor%
,  Semester student
   (Robert Dünner)
. ETH Zürich, Switzerland


  \margin{%
  2011
    --- 2012
}
Co-Lecturer%
,  English for nurses: an introduction to academic reading
. Zurich University of Applied Sciences, Switzerland


  \margin{%
  2007
    --- 2008
}
Tutor for undergraduate calculus%
. Indiana University Bloomington, IN, U.S.A.


  \margin{%
  2005
    --- 2007
}
Teaching assistant in physics%
. EPFL, Lausanne, Switzerland


  \margin{%
  2003
    --- 2004
}
Teaching assistant in scientific programming%
. EPFL, Lausanne, Switzerland



\section*{Selected presentations}
  \margin{Jul 2019}
ETH Zurich Food Biotechnology Seminar
(talk)
  {\condensed Zurich, Switzerland}


  \margin{Feb 2019}
Univeristy of Minnesota Seminar
(talk)
  {\condensed St. Paul, MN, USA}


  \margin{Aug 2018}
Penn State Microbiome Center Seminar
(talk)
  {\condensed State College, PA, USA}


  \margin{Aug 2018}
ISME Conference
(talk)
  {\condensed Leipzig, Germany}


  \margin{Jul 2018}
HFSP Fellows Meeting 2018
(poster)
  {\condensed Toronto, Canada}


  \margin{Jun 2018}
International Sourdough Symposium
(poster)
  {\condensed Cork, Ireland}


  \margin{Jun 2018}
NYU Genomics Symposium
(talk)
  {\condensed New York, NY, USA}


  \margin{Jan 2018}
MIT Ecology Meeting
(talk)
  {\condensed Cambridge, MA, USA}


  \margin{Nov 2017}
Workshop, Symbiosis in the microbial world: from ecology to genome evolution)
(talk)
  {\condensed West Sussex, UK}


  \margin{Jul 2017}
Gordon Research Conference: Microbial Population Biology
(poster)
  {\condensed Andover, NH, USA}


  \margin{Jul 2017}
Gordon Research Seminar: Microbial Population Biology
(talk)
  {\condensed Andover, NH, USA}


  \margin{Mar 2017}
Winter q-Bio Conference
(talk)
  {\condensed Kauai, HI, USA}


  \margin{Oct 2016}
Weizman Genome Evolution Conference
(talk)
  {\condensed Rehovot, Israel}


  \margin{Aug 2016}
ISME Conference
(poster)
  {\condensed Montreal, Canada}


  \margin{Aug 2015}
ESEB Conference
(talk)
  {\condensed Lausanne, Switzerland}


  \margin{Jul 2015}
SMBE Conference
(poster)
  {\condensed Vienna, Austria}


  \margin{May 2015}
HIV Dynamics and Evolution Conference
(talk)
  {\condensed Budapest, Hungary}


  \margin{Feb 2014}
New Zealand Phylodynamics Meeting
(talk)
  {\condensed Waiheke, New Zealand}


  \margin{Nov 2013}
Epidemics Conference
(poster)
  {\condensed Amsterdam, Netherlands}


  \margin{May 2013}
MCBE Conference
(poster)
  {\condensed Montpellier, France}


  \margin{May 2013}
HIV Dynamics and Evolution Conference
(poster)
  {\condensed Utrecht, Netherlands}


  \margin{Jul 2012}
Gordon Research Conference: Drug Resistance Evolution
(poster)
  {\condensed Easton, MA, USA}


  \margin{Jan 2012}
EE2 Workshop: Facing the challenge of infectious diseases
(poster)
  {\condensed Aosta, Italy}


  \margin{Jul 2011}
Gordon Research Conference: Microbial Population Biology
(poster)
  {\condensed Andover, NH, USA}


  \margin{Nov 2011}
EAWAG Aquatic Ecology and Macroevolution Seminar
(talk)
  {\condensed Kastanienbaum, Switzerland}



\section*{Other Acadmic Activities}

\subsubsection*{Grant Referee}
Swiss National Science Foundation
    
\subsubsection*{Journal Referee}
eLIFE; PNAS; Ecology Letters; Environmental Microbiology; Communications Biology; Proceedings of the Royal Society B; The American Naturalist; PLoS Computational Biology; Molecular Biology and Evolution; Journal of the Royal Soceity Interface; mSystems; Bioinformatics; Epidemics; Journal of Theoretical Biology; Theoretical Population Biology; Journal of Acq. Immune Deficiency Syndromes; Scientific Reports; PLoS ONE; Applied Mathematics and Computation; International Health

\subsection*{Seminar Organizing Committee}
\begin{itemize}[label={},leftmargin=0pt]
  \item Parson's Microbial Systems Seminar, Massachusetts Institute of Technology
  \item Ecology, Evolution, Environment, Behavior (E3B) Seminar, University of Zurich/ETH Zurich
  \item Zurich Interaction Seminar, University of Zurich/ETH Zurich
\end{itemize}

\subsection*{Other activities}
\begin{itemize}[label={},leftmargin=0pt]
  \item%
    Maintainer at Brewsci/bio.
    Bioinformatics formulae for the Linuxbrew and Homebrew package managers..
    \url{https://brewsci.github.io/homebrew-bio/}
\end{itemize}

\section*{References}
\begin{itemize}[label={},leftmargin=0pt]
  \item%
    Prof. Otto Cordero, Massachusetts Institute of Technology, \href{mailto:ottox@mit.edu}{ottox@mit.edu}
  \item%
    Prof. Sebastian Bonhoeffer, ETH Zurich, \href{mailto:seb@env.ethz.ch}{seb@env.ethz.ch}
  \item%
    Prof. Christophe Fraser, University of Oxford, \href{mailto:christophe.fraser@bdi.ox.ac.uk}{christophe.fraser@bdi.ox.ac.uk}
\end{itemize}

\end{document}
